\documentclass[twoside, 11pt]{article}

\usepackage[preprint]{jmlr2e}

\usepackage{natbib}

\usepackage{amsmath}
\usepackage{amsfonts}
\usepackage{mathtools}

% for bayesian network diagrams
% ref: https://github.com/jluttine/tikz-bayesnet
\usepackage{float}
\usepackage{tikz}
\usetikzlibrary{bayesnet}

% ensure sufficient marginspace for todos
\setlength {\marginparwidth }{2cm}
\usepackage[obeyFinal]{todonotes}
% \setuptodonotes{inline}

% define notation for norm and abs that scale nicely.
% ref: https://tex.stackexchange.com/a/297263
\let\oldnorm\norm
\let\norm\undefined
\DeclarePairedDelimiter\norm{\lVert}{\rVert}

\let\oldabs\abs
\let\abs\undefined
\DeclarePairedDelimiter\abs{\lvert}{\rvert}

\DeclarePairedDelimiter\card{\lvert}{\rvert}

\DeclareMathOperator*{\argmax}{arg\,max}
\DeclareMathOperator*{\argmin}{arg\,min}
\DeclareMathOperator*{\gammad}{Gamma}
\DeclareMathOperator*{\poissond}{Poisson}
\DeclareMathOperator*{\negbind}{Neg-bin}

\newcommand{\reals}[0] {\mathbb{R}}
\newcommand{\nonnegint}[0] {\mathbb{N}_{\geq 0}}

\newcommand{\E}[0] {\mathbb{E}} %expect

\newcommand{\bigO}[0] {\mathcal{O}}


\begin{document}

% style self-loop edges.
\tikzset{every loop/.style={min distance=10mm,in=60,out=120,looseness=10}}


\author{\name Reuben Fletcher-Costin}

\title{}

\maketitle

\begin{abstract}%
This paper considers a hybrid hidden Markov model for event counts with an additive Poisson noise term. 
\end{abstract}

\section{probabilistic model}
\subsection{Prelude: hidden Markov model}

Consider a first-order hidden Markov model (HMM) with a hidden state $x_t$ at each time $t=0,\ldots,T$ that generates observable evidence $z_t$ for $t=1,\ldots,T$. We use $z_{1:T}$ to denote $z_1, \ldots, z_T$. Due to the Markov assumptions, the joint distribution of $Z_{1:T}$ and $x_{0:T}$ factorises as
\begin{equation}
P(z_{1:T}, x_{0:T}) = P(x_0) \prod_{t=1, \ldots, T} P(z_t \mid x_t) P(x_t \mid x_{t-1} ) \; .
\end{equation}
The condition dependence structure of the HMM is illustrated in Figure~\ref{fig:hmm}.Each hidden state $x_t$ takes values in a finite state space $S = \{ s_1, \ldots, s_n \}$, and that each generated observation $z_t$ is an event count in $\nonnegint$. The probability of transitioning from a state $x_t=s_j$ at time $t$ to a state $x_{t+1}=s_i$ at time $t+1$ is given by
\begin{equation}
P(x_{t+1}=s_i \mid x_t=s_j) = A_{i,j}
\end{equation}
where $A$ is an $n \times n$ stochastic matrix. The probability of observing $z_t=k$ events at time $t$ is given by
\begin{equation}
P(z_t=k \mid x_t=s_j) = B_{k, j}
\end{equation}
where column $B_j$ is a distribution for $k$ over $\nonnegint$.

% standard hmm

\begin{figure}[H]
\tikz{
	% nodes
	\node[latent] (x) {$x_t$}; %
	\node[latent,left=of x] (x0) {$x_0$}; %
	\node[obs,below=of x] (z) {$z_t$}; %
	% plates
	\plate {timeplate} {(x) (z)} {$T$};%
	% edges
	\edge {x0} {x}
	\edge {x} {z}
	% self-loop edges
	\path[->] (x) edge [loop above] node {$t=1,\ldots,T$; $t-1 \mapsto t$} ();
}
\caption{Hidden Markov model}
\label{fig:hmm}
\end{figure}

\subsection{Hidden Markov model with additive Poisson noise}

Now consider a HMM where the event count $z_t$, generated from the hidden state $x_t$, is no longer directly observable. Instead, at each time $t$ we observe an event count $y_t \in \nonnegint$ such that
\begin{equation}
y_t = x_t + k_t
\end{equation}
where $k_t \in \nonnegint$ is generated from a Poisson distribution with a rate $\lambda > 0$ that does not vary over time. We assume a prior distribution $\gammad(\alpha, \beta)$ over $\lambda$. The joint probability distribution for this model is assumed to be
\begin{align}
& P(y_{1:T}, z_{1:T}, x_{0:T}, k_{1:t}, \lambda, \alpha, \beta) \nonumber \\
= & P(x_0) P(\alpha, \beta) P(\lambda \mid \alpha, \beta) \prod_{t=1, \ldots, T} P(y_t \mid k_t, z_t) P(k_t \mid \lambda) P(z_t \mid x_t) P(x_t \mid x_{t-1} ) \; .
\end{align}
This conditional dependence structure is shown in Figure~\ref{fig:hmmpoisson}.

The likelihood $P(y_t=y \mid x_t, \alpha, \beta)$ of observing $y_t=y$ events given the latent variables $x_t, \alpha, \beta$ reduces to a discrete convolution between the likelihood of the noise-free HMM $P(z_t=k \mid x_t)$ and the likelihood of the noise model $P(k_t=k \mid \alpha, \beta)$:
\begin{align}
& P(y_t=y \mid x_t, \alpha, \beta) \nonumber \\
= & \int_{(z, k, \lambda)} P(y_t=y \mid x_t, \alpha, \beta, z_t=z, k_t=k, \lambda=\lambda) P(z_t=z, k_t=k, \lambda=\lambda \mid x_t, \alpha, \beta) \, d(z, k, \lambda) \nonumber \\
= & \int_{(z, k, \lambda)}
P(y_t=y \mid z_t=z, k_t=k)
P(z_t=z \mid x_t)
P(k_t=k \mid \lambda)
P(\lambda \mid \alpha, \beta)
\, d(z, k, \lambda) \nonumber \\
= & \int_{(k, \lambda)}
P(z_t=y-k \mid x_t)
P(k_t=k \mid \lambda)
P(\lambda \mid \alpha, \beta)
\, d(k, \lambda) \nonumber \\
= & \sum_{k \in \nonnegint} P(z_t = y-k \mid x_t)
\int_{\lambda} P(k_t = k \mid \lambda) P(\lambda \mid \alpha, \beta) \, d \lambda \, .
\end{align}
The choice of $P(k_t = k \mid \lambda)$ as the Poisson distribution and $P(\lambda \mid \alpha, \beta)$ as its conjugate prior Gamma distribution leads to a closed-form expression for the integral:
\begin{align}
& \int_{\lambda}
P(k_t = k \mid \lambda) P(\lambda \mid \alpha, \beta)
\, d \lambda \nonumber \\
= & \int_{\lambda}
\poissond( k ; \lambda) \gammad( \lambda ; \alpha, \beta)
\, d \lambda \nonumber \\
= & \int_{\lambda}
\frac{\lambda^k e^{-\lambda}}{k!}
\frac{\beta^{\alpha}}{\Gamma(\alpha)}
\lambda^{\alpha-1} e^{-\beta \lambda}
\, d \lambda \nonumber \\
= & \int_{\lambda}
\frac{1}{k!}
\frac{\beta^{\alpha}}{\Gamma(\alpha)}
\lambda^{(\alpha+k)-1} e^{-(\beta+1) \lambda}
\, d \lambda \nonumber \\
= &
\frac{\Gamma(\alpha+k)}{\Gamma(\alpha) k!}
\frac{\beta^{\alpha}}{(\beta+1)^{\alpha+k}}
\int_{\lambda}
\frac{(\beta+1)^{\alpha+k}}{\Gamma(\alpha+k)}
\lambda^{(\alpha+k)-1} e^{-(\beta+1) \lambda}
\, d \lambda \nonumber \\
= &
\frac{\Gamma(\alpha+k)}{\Gamma(\alpha) k!}
\frac{\beta^{\alpha}}{(\beta+1)^{\alpha+k}}
\int_{\lambda}
\gammad(\lambda ; \alpha+k, \beta+1)
\, d \lambda \nonumber \\
= &
\frac{\Gamma(\alpha+k)}{\Gamma(\alpha) k!}
\frac{\beta^{\alpha}}{(\beta+1)^{\alpha+k}}
\nonumber \\
= &
\binom{\alpha-1+k}{k} \left( \frac{\beta}{\beta+1} \right)^{\alpha} \left( \frac{1}{\beta+1} \right)^{k} \, .
\end{align}
where we note that $\gammad(\lambda ; \alpha+k, \beta+1)$ is a density over $\lambda$ and hence integrates to unity. The final expression is known~\citep{gelman2013bayesian} to be the negative binomial density $k \sim \negbind\left(\alpha, \beta\right)$. Combining the above two derivations produces a terse expression for the likelihood
\begin{align}
P(y_t=y \mid x_t, \alpha, \beta)
= & \sum_{k \in \nonnegint} P(z_t = y-k \mid x_t) \negbind\left(k ; \alpha, \beta\right) \, \nonumber \\
= & \left[ P(z_t=k \mid x_t) * \negbind(k ; \alpha, \beta) \right] \Bigr \rvert_{k=y} \, .
\end{align}

% hmm with additive Poisson noise

\begin{figure}
\tikz{
	% nodes
	\node[obs] (y) {$y_t$}; %
	\node[latent,above=of y] (z) {$z_t$}; %
	\node[latent,above left=of y] (k) {$k_t$}; %
	\node[latent,above=of z] (x) {$x_t$}; %
	\node[latent,left=of x, xshift=-1cm] (x0) {$x_0$}; %

	\node[latent,left=of k] (lambda) {$\lambda$}; %
	\node[latent,left=of lambda] (alphabeta) {$\alpha, \beta$}; %
	% plates
	\plate {timeplate} {(x) (z) (k) (y)} {$T$};%
	% edges
	\edge {x0} {x}
	\edge {x} {z}
	\edge {z} {y}
	\edge {k} {y}
	\edge {lambda} {k}
	\edge {alphabeta} {lambda}
	% self-loop edges
	\path[->] (x) edge [loop above] node {$t=1,\ldots,T$; $t-1 \mapsto t$} ();
}
\caption{Hidden Markov model with Poisson noise}
\label{fig:hmmpoisson}
\end{figure}

\subsection{Standard HMM inference tasks}

As explained by \citet{rabiner1989tutorial}, \citet*{russell2002artificial}, given a hidden Markov model, there are a number of standard inference tasks we may wish to perform:

\begin{enumerate}
\item ``filtering'': estimating the posterior distribution at the current time $t=T$ over latent variables given evidence to date $y_{1:t}$
\item ``smoothing'': estimating the posterior distribution over latent variables given the evience to date at some historical time $t<T$
\item ``prediction'': estimating the posterior distribution over latent variables at future times $t>T$ given evidence to date
\item most likely explanation: recovering a trajectory of latent states that is best, in some particular sense (e.g. via the Viterbi algorithm)
\item estimating or training the parameters of our Markov model (E.g. via the Baum-Welch algorithm, an instance of Expectation Maximisation algorithms).
\end{enumerate}

We will focus on the first task, ``filtering''.

\subsection{Toward a forward algorithm}

Here's the rough plan of attack:
\begin{enumerate}
\item follow \citet*{russell2002artificial} derivation of the forward update equation for the posterior
\item make the working assumption that we can approximate the posterior distribution with $\gammad$ distributions, one for each hidden state $s \in S$ in the discrete state space. give a brief example illustrating why we need a way to encode belief of varying rates $\lambda$ for different states in the discrete space
\item note how this family of approximations are not closed under the transition operator on $S$ -- whenever we apply the transition operator we end up with a mixture of $\gammad$ distributions. Similarly for conditioning on evidence.
\item look for a best approximation of the resulting mixture within our approximation space, in the sense of minimal $KL$-divergence
\item derive closed-form equations for our approximation to project each mixture back down to a single $\gammad$ distribution
\item comment about approximation error
\end{enumerate}

The goal is to derive a computationally tractable way to update the posterior of the latent variables $(x_t, \lambda)$ given the information from one new observation of an event count $y_{t+1}$. An exact update algorithm would be some function $F$ of the form:
\begin{equation}
P(x_{t+1}, \lambda \mid y_{1:t+1}) = F\left( P(x_t, \lambda \mid y_{1:t}), y_{t+1}\right)
\end{equation}
One complication impeding exact computation is the coupling between the latent variables $\lambda$ and $x_{t+1}$ after conditioning on $y_{t+1}$. For a simple example of this, consider a state space $S$ consisting of two hidden states $s_1$ and $s_2$ equipped with observation models $P(z_t \mid x_t=s_1)$ and $P(z_t \mid x_t=s_2)$ with $\E \left[ P(z_t \mid x_t=s_1) \right] = 1$ and $\E \left[ P(z_t \mid x_t=s_2) \right] = 2$. Assume transitions between $s_1$ and $s_2$ are impossible, but that we have very little prior knowledge about the magnitude of $\lambda$, the rate of the Poisson noise, or the initial state $x_0$. Suppose we observe event data $y_1, \ldots, y_T$ over a period of time, and learn that the average event count $\sum_{t=1}^T y_t \approx 3$.  Then if the system were dwelling in state $x_t=s_1$ we would infer that $\lambda \approx 2$, and alternatively if system had occupied state $x_t=s_2$ we would infer that $\lambda \approx 1$, in order for the expected rate of ``signal'' event counts $Z_t$ and ``noise'' event counts $k_t$ to sum to give the observed event counts $y_t \approx 3$. This means that an accurate representation of our posterior belief needs to be rich enough to model coupling between $\lambda$ and $x_{t+1}$.

More abstractly, we can think of updating the posterior as $\phi_{t+1} = F \left( \phi_t, y_{t+1} \right)$ where the $\phi$ are elements of some space $\Phi$ that is closed under application of the following operators:
\begin{enumerate}
\item a transition operator $T : \Phi \rightarrow \Phi$ that evolves the latent states forward one timestep.
\item a condition-on-observation operator $O_y : \Phi \rightarrow \Phi$, where $y$ is a new observation.
\end{enumerate}
If we have $T$ and $\{O_y\}_{y \in \nonnegint}$ then we can define $F(\Phi, y_{t+1}) = O_{y_{t+1}} \circ T (\Phi)$. Instead of representing the posterior exactly, we can select some approximation space $\Phi$ such that $P(x_t, \lambda \mid y_{1:t} ) \approx \phi_t \in \Phi$.

\subsection{Discrete approximation of rate parameter $\lambda$}

One simple approximation scheme is to fix a-priori $m$ different values of the noise rate parameter $\{ \lambda_1, \ldots, \lambda_m \} = \Lambda_m \subset \reals_{>0}$. Then, if the state space $S$ contains $n$ hidden states $s_1, \ldots, s_n$, we can reduce the model to a HMM over the $n \times m$ discrete state space $(s_i , \lambda_j ) \in S \times \Lambda_m$. One way to construct this is to consider $m$ trivial single-state HMMs for each value of $\lambda_j$, then combine them into a single HMM with state space $\{ \lambda_1, \ldots, \lambda_m \}$ but no possible transitions between distinct states by taking the disjoint union of their state spaces $\{ \lambda_j \}_j$, then take the product of the resulting HMM with the standard noise-free discrete HMM for our states $x_t \in S$.  A downside to this approach is that if we pick too few values of $m$ we may get a poor approximation, but as we pick more values of $m$, the cumulative computational effort to repeatedly apply the transition operator over $t = 1, \ldots, T$ while updating the approximate posterior is $ \bigO (T n^2 m ) $.

\subsection{Factored approximation with Gamma distributions}

The simple discrete approximation scheme described in the previous subsection required us to store $m$ non-negative values $P(x_t=s_i, \lambda=\lambda_j \mid y_t)$ for each $i$, $t$ in order to approximate the distribution $\lambda \sim P(x_t=s_i, \lambda \mid y_t)$. We may be able to approximate the distribution of $\lambda$ for fixed $x_t=s_i$, $t$ more compactly with an appropriately chosen parametrised family of distributions. One obvious choice to investigate is the conjugate prior distribution for $\lambda$, $\gammad(\alpha, \beta)$. This suggests the following approximation:
\begin{equation}
P\left(x_t=s_i, \lambda \mid y_{1:t} \right) \approx \phi(x_t=s_i, \lambda) \coloneqq g(\lambda ; \alpha_{i,t}, \beta_{i,t}) P(x_t=s_i \mid y_{1:t} ) ,
\end{equation}
where $g(\lambda ; \alpha_{i,t}, \beta_{i,t})$ is the probability density function of the $\gammad(\alpha_{i, t}, \beta_{i, t})$ distribution and we regard $g(\lambda ; \alpha_{i,t}, \beta_{i,t}) \approx P(\lambda | x_t=s_i, y_{1:t})$. With this approximation, we need only maintain three scalar values for each $i$, $t$:
\begin{enumerate}
\item $P(x_t=s_1 \mid y_{1:t})$, the value of the marginal distribution,
\item $\alpha_{i,t}>0$, a shape parameter, and
\item $\beta_{i,t}>0$ a rate parameter.
\end{enumerate}

Formally, we define our approximation space $\Phi$ as
\begin{equation}
\Phi \coloneqq \left\{ \phi_{\alpha, \beta, c} : \alpha = (\alpha_s)_s, \beta = (\beta_s)_s, c = (c_s)_s, \alpha_s > 0, \beta_s > 0, c_s \geq 0, \sum_{s \in S} c_s = 1 \right\}
\end{equation}
where each element $\phi_{\alpha, \beta, c} \in \Phi$ is a function $\phi_{\alpha, \beta, c} : \reals_{>0} \times S \rightarrow \reals_{\geq 0}$ that acts on a rate-state pair $(\lambda, s)$ by
\begin{equation}
\phi_{\alpha, \beta, c} (\lambda, s) \coloneqq c_s \; g(\lambda ; \alpha_s, \beta_s) .
\end{equation}

This approximation scheme compresses the storage requirements per time $t$ from $n m$ scalar values of the previous approximation scheme down to $3 n$ scalar values. This reduction in parameters comes at the cost of no longer being able to represent multi-modal behaviour of $\lambda \sim P(\lambda | x_t=s_i, y_{1:t})$. Note that we structurally retain the ability to represent different noise rates $\lambda$ coupled with different hidden states $s \in S$. We are not directly concerned about storage requirements, but a smaller state space hints at the possibility of reduced computational effort.


\bibliography{note}

\end{document}
